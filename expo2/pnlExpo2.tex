%%%%%%%%%%%%%%%%%%%%%%%%%%%%%%%%%%%%%%%%%%%%%%%%%%%%%%%%%%%%%%%%%%%%%%%%%%%%%%%%%%%%%
% 			Facultad de Ciencias, UAEM.  		Octubre de 2014
% 
%	Alumno: 			Emanuel García Pérez
%	Asginatura:		Procesamiento del Lneguaje Natural
%	Proyecto:		Exposición - Artículo
%	Tema:			"Clasificación de polaridad en textos con opiniones
%					 en español mediante análisis sintáctico de dependencias"
%
%%%%%%%%%%%%%%%%%%%%%%%%%%%%%%%%%%%%%%%%%%%%%%%%%%%%%%%%%%%%%%%%%%%%%%%%%%%%%%%%%%%%%
 

\documentclass{beamer}
 
%\usepackage[spanish,activeacute]{babel}
\usepackage[latin1]{inputenc}
\usepackage{beamerthemeshadow}
\usepackage{graphicx}

\title{\textbf{Clasificaci\'on de polaridad en textos con opiniones en espa\~nol mediante an\'alisis sint\'actico de dependencias}} 
\author{Emanuel Garc\'ia P\'erez}
\date{\today}

\begin{document}

\frame[allowframebreaks]{\titlepage}
\section[Contenidos]{}
\frame{
\transdissolve[duration=0.2]
\tableofcontents
} 

\section{INTRODUCCI\'ON}
\frame{
\transdissolve[duration=0.2]
\frametitle{}
\begin{itemize}
	\item En los \'ultimos a\~nos el alcancen de los blogs, foros y redes sociales han hecho que millones de usuarios los usen para expresar sus opiniones sobre diversos temas.
	\item Esta gran variedad de opiniones y cr\'iticas que abundan en la web son de gran utilidad para que vendedores y fabricantes conozcan el impacto de sus productos en los consumidores.
\end{itemize}
\begin{figure}
  \centering
    \includegraphics[width=0.5\textwidth]{sentiment1.jpg}
  \label{fig:ejemplo}
\end{figure}
}

\frame{
\transdissolve[duration=0.2]
\frametitle{}
Debido a las ventajas que presenta toda esa informaci\'on y la complejidad que conlleva su an\'alisis es necesario la b\'usqueda de soluciones eficientes capaces de monitorizar este flujo de informaci\'on, lo cual ha contribuido al estudio y desarrollo de la miner\'ia de opiniones(MO). La MO de enfoca en el tratamiento autom\'atico de informaci\'on con opini\'on, lo que permite extraer la polaridad de un texto.
}


\section{ESTADO DEL ARTE}
\section{CLASIFICACI\'ON DE OPINIONES BASADA EN DEPENDENCIAS}
\subsection{Propuesta Base}
\subsection{Tratamiento de la Intensificaci\'on}
\subsection{Tratamiento de las oraciones adversativas}
\subsection{Tratamiento de la Negaci\'on}
\subsubsection{Identificaci\'on del alcance de la negaci\'on}
\subsubsection{Modificaci\'on de la polaridad}
\section{RESULTADOS EXPERIMENTALES}
\subsection{Implementaci\'on}
\subsection{Evaluaci\'on}
\section{CONCLUSIONES Y TRABAJO FUTURO}

\section{BIBLIOGRAF\'IA}
\frame{
\transdissolve[duration=0.2]
\frametitle{Bibliograf\'ia}
\begin{thebibliography}{99}
\bibitem{}Desarrollo de un Sistema de Recuperaci\'on de Informaci\'on para Publicaciones
Cient\'ificas del \'Area de Ciencias de la Computaci\'on. Kuna, H., Rey, M., Martini, E., Solonezen, L., Podkowa. Revista Latinoamericana de Ingenier\'ia de Software 2014.
\bibitem{}THE AR-INDEX: COMPLEMENTING THE H-INDEX. Bihui Jin. National Library of Science CAS, Beijing, 100080, China. 2005-2007.
\bibitem{}http://www.core.edu.au/
\end{thebibliography}
}

\end{document}