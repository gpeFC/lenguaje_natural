%%%%%%%%%%%%%%%%%%%%%%%%%%%%%%%%%%%%%%%%%%%%%%%%%%%%%%%%%%%%%%%%%%%%%%%%%%%%%%%%%%%%%
% 			Facultad de Ciencias, UAEM.  		Noviembre - 2014
% 
%	Alumno: 			Emanuel García Pérez
%	Asginatura:		Procesamiento del Lenguaje Natural
%	Proyecto:		Exposición - Artículo
%	Tema:			"Agentes Inteligentes: Recuperación Autónoma
%					 de Información en la Web"
%
%%%%%%%%%%%%%%%%%%%%%%%%%%%%%%%%%%%%%%%%%%%%%%%%%%%%%%%%%%%%%%%%%%%%%%%%%%%%%%%%%%%%%
 

\documentclass{beamer}
 
%\usepackage[spanish,activeacute]{babel}
\usepackage[latin1]{inputenc}
\usepackage{beamerthemeshadow}
\usepackage{graphicx}

\title{\textbf{Agentes Inteligentes: Recuperaci\'on Aut\'onoma de Informaci\'on en la Web}} 
\author{Emanuel Garc\'ia P\'erez}
\date{\today}

\begin{document}

\frame[allowframebreaks]{\titlepage}
\section[Contenidos]{}
\frame{
\transdissolve[duration=0.2]
\tableofcontents
} 



\section{INTRODUCCI\'ON}
\frame{
\transdissolve[duration=0.2]
\frametitle{La web como fuente de informaci\'on}
El constante crecimiento de Internet durante los \'ultimos a\~nos ha sido, dentro del campo de la informaci\'on, uno de los desarrollos m\'as importantes. Respecto al \'ambito cient\'ifico, muchas de las fuentes de informaci\'on tradicionales ya se encuentran en la red, lo que ha propiciado poner en evidencia el problema concerniente a la \textbf{recuperaci\'on de informaci\'on} en la web.
}
\frame{
\transdissolve[duration=0.2]
\frametitle{Recuperaci\'on de informaci\'on en la web}
Actualmente los sistemas de recuperaci\'on de informaci\'on en la web utilzian dos mecanismos, no excluyentes entre si y que se pueden combinar, para solventar este problema:\\
\begin{itemize}
	\item \textbf{B\'usqueda mediante palabras clave} Se pueden aplicar t\'ecnicas que mejoren los resultados; tesauros de similitud, an\'alisis de cluster, utilizaci\'on de redes neuronales.
	\item \textbf{Clasificaci\'on de p\'aginas en categor\'ias} La clasificaci\'on suele realizarse manualmente, aunque tambi\'en se ha realizado de forma autom\'atica utilziando mapas autoorganizados para establecer las categor\'ias de los t\'erminos, los cuales ser\'an empleados para definir vectores de p\'aginas web.
\end{itemize}
}
\frame{
\transdissolve[duration=0.2]
\frametitle{Sistemas generales de b\'usqueda}
Ya sea que se utilice una b\'usqueda por palabras clave o a trav\'es de la clasificaci\'on previa de p\'aginas, o una combinaci\'on de ambos, estos m\'etodos parten de la existencia de una \textbf{base de datos}, de un tama\~no considerablemente grande, que contenga la colecci\'on de p\'aginas web, lo que conlleva a que la experiencia de los usuarios se vea mermada por la \textbf{baja presici\'on} de las respuestas devueltas y la enorme cantidad de resultados asociados a una b\'usqueda particular. \\
Es debido a este tipo de problemas que se pretende obtener resultados m\'as precisos, no necesariamente en gran cantidad, bas\'andose en el uso de \textbf{agentes inteligentes}.
}



\section{AGENTES INTELIGENTES}
\frame{
\transdissolve[duration=0.2]
\frametitle{Agentes Inteligentes}
\begin{figure}
  \centering
    \includegraphics[width=0.9\textwidth]{agentes.jpg}
  \label{fig:ejemplo}
\end{figure}
}



\section{AGENTES INTELIGENTES Y RECUPERACI\'ON DE INFORMACI\'ON EN LA WEB}
\frame{
\transdissolve[duration=0.2]
\frametitle{}
}


\subsection{La elecci\'on de los puntos de partida}
\frame{
\transdissolve[duration=0.2]
\frametitle{}
}


\subsection{Activaci\'on de enlaces}
\frame{
\transdissolve[duration=0.2]
\frametitle{}
}


\subsection{Selecci\'on de p\'aginas por contenido}
\frame{
\transdissolve[duration=0.2]
\frametitle{}
}

\subsubsection{T\'ecnicas de recuperaci\'on de informaci\'on}
\frame{
\transdissolve[duration=0.2]
\frametitle{}
}

\subsubsection{Estudio de enlaces}
\frame{
\transdissolve[duration=0.2]
\frametitle{}
}



\section{CONCLUSIONES}
\frame{
\transdissolve[duration=0.2]
\frametitle{}
}



\end{document}